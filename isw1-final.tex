\documentclass[a4paper,spanish]{article}

\usepackage[spanish,activeacute]{babel}
\usepackage{moreverb}
\usepackage{fancyhdr}
\usepackage{graphicx}
\usepackage{multicol}

\oddsidemargin 0in
\textwidth 6.2in
\topmargin 0in
\addtolength{\topmargin}{-.5in}
\textheight 10in
\parskip=1ex
\pagestyle{fancy}
%usar el segundo nivel de enumeracion con letras
%\renewcommand{\labelenumii}{\alph{enumii}. }

\newcommand{\rev}{\marginpar{REVISAR}}
\newcommand{\nohecho}{\marginpar{NO HECHO}}

%espaciado
\newcommand{\vsp}{\vspace{0.4cm}}
\newcommand{\hsp}{\hspace{0.12cm}}

%comandos para el resumen
\newcommand{\tab}{\hspace*{1cm}}
\newcommand{\defi}[2]{\bfseries #1: \mdseries #2}
\newcommand{\ldefi}[1]{\bfseries #1: \mdseries}
\newcommand{\sndefi}[1]{\newline \tab \bfseries #1 \mdseries}
\newcommand{\sdefi}[2]{\newline \tab \bfseries #1: \mdseries #2}
\newcommand{\ssdefi}[2]{\newline \tab\tab \bfseries #1: \mdseries #2}
\newcommand{\sssdefi}[2]{\newline \tab\tab\tab \bfseries #1: \mdseries #2}
\newcommand{\subi}{\newline \tab \ensuremath{\bullet} }
\newcommand{\ssubi}{\newline \tab\tab \ensuremath{\bullet} }
\newcommand{\llamada}[1]{\begin{center} \bfseries #1 \mdseries \end{center}}
\newcommand{\nota}[1]{\vsp\defi{Nota}{#1}\vsp}


%noindent en todos lados
\parindent=0in 

\lhead{Ingenier\'{i}a de Software 1}
\rhead{Apunte de repaso general}

\cfoot{$\thepage$ de \pageref{theend}}

\begin{document}

Disclaimer: Este apunte no es autocontenido y fue pensado como un repaso de los conceptos, 
no para aprenderlos de aqu'i directamente.
%\begin{multicols}{2}
%\tableofcontents
%\end{multicols}
%\newpage

\section{Michael Jackson}

\subsection{Dominio de aplicaci'on}

\defi{Problema de desarrollo}{Problema de construir una m'aquina (ej: word=m'aquina de escribir)}

\defi{Dominio de aplicaci'on}{Parte de mundo en el que la m'aquina se hace sentir $\Rightarrow$ lugar en el
que es evaluada/aprobada}

Identificarlo es necesario para enfocarse en los requerimientos. Puede incluir cosas concrectas y 
abstractas (reglas de negocios). Hay que prestarle mucha atenci'on, entenderlo \textbf{es} entender 
lo que hay que hacer. Da la base para elegir apropiadamente el \textbf{marco} y los \textbf{m'etodos}
para la soluci'on.

Ej: Puede haber mecanismos preexistentes, que pueden tener que ser incorporados, modificados o
eliminados, y es necesario entenderlos para hacerlo.

El dominio de aplicaci'on es el material para trabajar, un \textbf{requerimiento} es qu'e hacer
con 'el.

Se debe diferenciar el dominio de aplicaci'on de la m'aquina a construir (es dificil en muchos casos).

\subsection{Dominios}

\ldefi{Dominios}
\sndefi{Est'aticos}
\sdefi{Din'amicos}{El tiempo es una dimensi'on en la cual las cosas cambian. Noci'on de \textit{estado}.}
\ssdefi{Inertes}{El estado solo cambia a las 'ordenes del usuario (procesador de texto)}
\ssdefi{Reactivos}{El estado cambia seg'un sus propias reglas, pero s'olo reaccionando ante una 'orden (base de datos)}
\ssdefi{Activos}{El estado cambia todo el tiempo por s'i mismo. (sistema operativo)}
\sssdefi{Aut'onomo}{No controlable (personas de la calle). Considerar cambios la m'aquina no va a medir.}
\sssdefi{Biddable}{Controlable por sugerencia (usuarios posibles de instruir)}
\sssdefi{Programable}{Controlable totalmente. En general se convierte en parte de la m'aquina} \\
\tab\tab Estas categorizaciones depende mucho de los l'imites que se establezcan para el dominio.

Ej: Estados de una construcci'on vial, en c/u hay que desviar el tr'afico en forma diferente, pero
las rutas en general son est'aticas para otros procesos (deliveries).

Para ver un dominio din'amico como est'atico saco una foto.

\defi{Unidimensional vs multidimensional}{Comunicaci'on punto a punto, texto vs grafos, imagen, teleconferencia}

\defi{Tangibles}{De mundo real. Inherentemente dificiles de describir porque son informales. Introducen reglas
de tiempo real.} \\
\defi{Intangibles}{Abstractos. Son casi siempre parte de la m'aquina a construir}

\defi{Interacciones de dominio}{Dividimos el dominio de aplicaci'on en varios dominios para modularizar,
pero s'olo si los \textbf{fen'omenos} de cada dominio estan muy o enteramente separados.}

\defi{Fen'omenos compartidos}{Lugar de interacci'on de los dominios (intersecci'on). Los fen'omenos compartidos
limitan la independencia de cada dominio, pero esta limitaci'on debe ser peque~na en comparaci'on con el tama~no
del dominio para que la partici'on tenga sentido.}

\defi{M'aquina}{Interact'ua con algunos dominios de la partici'on. Debe conectar el grafo de interacci'on
de los dominios (un dominio que no tiene un camino de interacci'on hacia la m'aquina indica un error en el
modelo).}

Se debe entender cada dominio y las interacciones (\textbf{fen'omenos compartidos}) entre 'estos y de 'estos
con la m'aquina.

\defi{Contexto del problema}{Dominio de aplicaci'on $\cup$ M'aquina $= \{D_1,...,D_n\} \cup$ M'aquina donde 
los $D_i$ son los dominios en los que se parte el dominion de aplicaci'on}

\defi{Prinicipio de relevancia de dominio}{Todo lo que es relevante a un \textbf{requerimiento} debe aparecer
en alguna parte del dominio de aplicaci'on. Esto implica que haya partes importantes del dominio de aplicaci'on
no conectadas directamente a la m'aquina. Este principio se usa para establecer la partici'on como los 
sustantivos importantes que aparezcan en los \textbf{requerimientos}.}

\subsection{Requerimientos}

La ingenier'ia de software tradicionalmente solo distingu'ia entre dentro y fuera de la m'aquina. Esto era un
problema porque el cliente (eventualmente) sabe poco de software y el equipo de desarrollo poco del negocio
del cliente. Para poder definir los \textbf{requerimientos} se precisa \textbf{incluir el dominio de
aplicaci'on dentro del espectro del problema}.

La m'aquina puede asegurar los requerimientos solo a trav'es de los \textbf{fen'omenos compartidos} con
los dominios $\in$ dominio de aplicaci'on. \\
\tab Un evento compartido sucede en 2 dominios 'o en un dominio y la m'aquina, aunque de cada lado se
ve distinto (de un lado la suba de temperatura y del otro la se~nal el'ectrica del sensor).

Gap entre requerimientos y lo que la m'aquina puede asegurar: Es producido por el hecho de que los requerimientos
no est'an formulados exclusivamente en t'erminos de fen'omenos compartidos y las cosas que pasan en la parte de
``afuera'' del dominio de aplicaci'on no es controlable directamente por la m'aquina (e incluso a veces 
incontrolable). Ej: El requerimiento dice que el usuario escribe una carta, pero 'este se niega a usar el teclado.

\defi{Especificaci'on}{Fen'omenos compartidos (la intersecci'on) entre el dominio de 
aplicaci'on y la m'aquina. Se derivan de los requerimientos, que son un subconjunto del dominio de
aplicaci'on.}

La idea es que entre la computadora y el programa (la m'aquina) impliquen la especificaci'on que a su
vez implica la satisfacci'on de requerimientos (idealmente, ya que los requerimientos en muchos casos
pertenecen al mundo informal y pueden tener errores).

\subsection{Especificaciones}

\defi{Especificaci'on}{Como fen'omenos compartidos, no necesariamente tienen sentido en el entorno
(dominio de aplicaci'on) porque \textbf{no son requerimientos} sino que fueron derivados de ellos.}

Las especificaciones no pueden ser solamente abstractas porque al pertenecer a la intersecci'on pueden
contener cosas del dominio de aplicaci'on, o sea, del mundo real. Si la especificaci'on es solo abstracta,
esto implicar'a que el problema que se est'a atacando sea abstracto.

\section{No silver bullet}

\ldefi{Problemas de la ingenier'ia de software}
\sdefi{Esencia}{Problemas inherentes}
\sdefi{Accidente}{Problemas coyunturales}

\defi{Programa}{Es una representaci'on conceptual, lo dificil es especificarla, dise~narla y testearla, 
no programarla.}

\ldefi{Problemas esenciales}
\sdefi{Complejidad}{Todas las partes son diferentes (si son iguales, se hacen comportamiento compartido)
lo cual redunda en un enorme n'umero de estados (esto hace dificil el testing y los controles como el de 
seguridad).}
\sdefi{Conformidad}{Hay que lidiar con muchas cosas diferentes (reglas, procesos existentes) hechas por 
diferentes personas o instituciones a lo largo de mucho tiempo, lo que provoca que no necesariamente 
hay una armon'ia entre ellas. Otras disciplinas (f'isica) buscan una armon'ia que puede ser dificil 
de encontrar, pero al menos existente.}
\sdefi{Cambiabilidad}{Al existir la posibilidad de hacer cambios baratos (en el sentido material) 'estos
son exigidos, a diferencia de en otras industrias que el producto entero es reemplazado por uno nuevo.
El programa tiene que ser configurable, y adaptable como m'inimo al hardware sobre el cual est'a corriendo,
que cambia muy r'apidamente.}
\sdefi{Invisibilidad}{La inherente realidad abstracta del software hace que sea m'as dificil hacer un 
diagrama que lo represente o, al hacerlo, el gap entre diagrama y realidad es mayor que entre un plano
y el edificio.}

\defi{M'etodos anteriores}{(atacaron accidentes, no escencia, por lo que su ganancia es limitada)}
\sdefi{Lenguajes de alto nivel}{Solo mejoran el desarrollo de la m'aquina}
\sdefi{Multitarea}{Solo mejora turnaround}
\sdefi{Unificaci'on de herramientas}{Solo mejora el desarrollo}

Lo dificil acerca de la ingenier'ia de software es decidir que se quiere decir, no decirlo. 

La inteligencia artificial no sirve de todo, porque para hacer un sistema experto es necesario tener un
experto humano previamente.

Programaci'on autom'atica es equivalente a m'as alto nivel, siempre se debe escribir el input 
(la especificaci'on).

Lo importante es debuguear la especificacion, asi que los verificadores autom'aticos no sirven.

\section{M\'{a}quinas de estado finitas (FSMs)}

\defi{FSM}{Nodos y transiciones con label}

\defi{Deadlock}{Estado sin trancisiones saliendo de 'el. No necesariamente implica un mal modelado
(puede indicar una finalizaci'on del proceso).}

\defi{Composici'on paralela $A||B$}{Los nodos son pares ordenados de estados. Se avanza en la coordenada
necesaria cuando uno avanza para las etiquetas no compartidas o cuando \textbf{ambos} avanzan en las
etiquetas compartidas. Esto sirve para ``comunicar'' distintas FSMs.}

\ldefi{Extensiones} \\
\tab Agregar a las etiquetas \textbf{condiciones} o \textbf{acciones} sobre variables. \\
\tab Agregar a los nodos condiciones (solo se puede reposar en ellos si se cumple la condici'on).

\defi{Temporizadas}{Se agrega el tipo de variable \texttt{Timer}}

Se puede decidir hacerlas no determin'isticas.

Las FSMs especifican aspectos din'amicos (el mapa conceptual solo est'aticos). Se pueden bindear al modelo
y luego usar OCL.

\section{Descripciones y an\'{a}lisis de software}

\defi{Ingenier'ia de software}{Aplicaci'on del conocimiento cient'ifico al proceso de desarrollo.}

\defi{Proceso de desarrollo}{Forma de transformar requerimientos en productos de software.}

\defi{Calidad}{Depende del punto de vista (programador, usuario, manager)}

No silver bullet, pero buenas pr'acticas probadas mejoran el funcionamiento.

\defi{Requerimientos}{Funcionales y no funcionales, 'estos 'ultimos pueden ser sobre el producto
(seguridad, performance) o sobre el proceso (tiempos de entrega, modalidad de trabajo)}.

El usuario no siempre necesita lo que quiere, se debe intentar mantener el sistema lo mas simple posible. \\
\tab Definir prioridad para los requerimientos (esencial, importante, deseable).

\defi{An'alisis de escenarios}{Ejemplos de sesiones de interacciones. Criterios de aprobaci'on para testing.}
\sdefi{Lenguajes}{Casos de uso, diagramas de secuencia, diagramas de actividad, castellano} \\
\tab \textbf{no} son requerimientos, los requerimientos se derivan de ellos.

\defi{Casos de uso}{Actores, descripci'on en castellano pero con estructura}
\sdefi{Actor}{Todo usuario o sistema que interactua con la ma'quina. Una persona en particular puede ser
varios actores (tipos de usuario).} \\
\defi{Diagrama de casos de uso}{Relaci'on entre casos (usa, extiende). Modularizaci'on, D\&C, cada diagrama
es interpretable independientemente.}

Los casos de uso son disparados por un actor o por tiempo. Se pueden ir desgranando incrementalmente.
Son los servicios a los usuarios y sirven para guiar el desarrollo (dise~no, implementaci'on y testing).
Forma adecuada de generar modelo que son los requerimientos. Los requerimientos funcionales son un caso
de uso, los no funcionales suelen asociarse a alguno(s).

Otros documentos: Glosario, prototipo de interfaz.

\defi{Diagrama de colaboraci'on}{Describe interacciones entre instancia de actor y de caso de uso}

\defi{Perfiles de trabajo}{Analista (director), especificador de casos de uso, dise~nador de GUI, 
arquitecto}

Pasos del proceso de modelado:
\begin{enumerate}
\item Identificar actores y casos de uso (t'itulos)
\item Asignar prioridades a casos de uso
\item Detallar casos de uso (precondici'on, postcondici'on, caminos alternativos y prohibidos). Incluye interfase
con actores y recursos.
\item Prototipar interfase de usuario
\item Hacer modelo de casos de uso (ayuda a generalizar casos de uso)
\end{enumerate}

\defi{Rastro de requerimientos}{Al cambiar los requerimientos se analizan los casos de uso afectados en el modelo 
y se ven que partes son afectadas del dise~no y la implementaci'on}
\sdefi{Matriz de trazabilidad}{Relaci'on de cada caso de uso con su impacto. En las columnas aparecen los distintos
aspectos del sistema (m'odulos, programas, recursos, procesos) y las filas los casos.}

\defi{Cuantificaci'on}{Vol'umen afecta soluci'on tecnol'ogica posible.}

\defi{Confiabilidad}{Se debe definir una m'etrica, siempre asumiendo uso aceptable}

\defi{Restricciones}{Externas (mundo real), internas (reglas del cliente) y de interfaz (con otros sistemas).}


\section{Arquitecturas de software}





\section{Res'umen de documentos}

\begin{multicols}{2}
\begin{itemize}
\item Modelo conceptual
\item Caso de uso
\item Diagrama de casos de uso
\item Diagrama de colaboraci'on
\item Glosario
\item Prototipo de interfaz
\item M'aquina de estado finita (FSM)
\item Diagrama de secuencia
\item Diagrama de actividad
\item Diagrama de flujo de datos (DFD)
\item Diagrama de entidad relaci'on (DER)
\item Diagrama de flujo de trabajo (workflow)
\end{itemize}
\end{multicols}

\label{theend}
\end{document}
